% !TeX spellcheck = de-DE
%-------------------------------------------------------------------------------
%   LaTeX Vorlage für Wissenschaftliche Arbeiten an der Fachhochschule Erfurt
%   Datum:   						2017-09-21
%   Autor:   						Stephan Rothe
%		Weiterentwicklung: 	Anja Haußen
%   email:   						anja.haussen@fh-erfurt.de
%-------------------------------------------------------------------------------

%--------------------------------------
%   Header
%--------------------------------------
\documentclass[
		abstract=false,
		appendixprefix=true,		  
		a4paper,                  % Papierformat A4
		10pt,                     % Schriftgröße 10pt
		headings=normal,          % kleinere Überschriften verwenden
		chapterprefix=false,      % Einfügen von "Anhang" bzw. "Kapitel" in Überschrift
		oneside,
		openright,                % einseitiges Layout
		titlepage,                % Titleseite verwenden
		listof=totoc,             % alle Listen in das Inhaltsverzeichnis
		headsepline,              % Trennlinie zum Seitenkopf Bereich headings
		plainheadsepline,         % Trennlinie zum Seitenkopf Bereich Plain
		bibliography=totoc,       % das Literaturverzeichnis in den TOC
		parskip=half-,						% Abstand nach Absatz
		numbers=noenddot					% damit hinter der letzten Ziffer kein Punkt steht (Kapitelnummerierung)
	]{scrreprt}                 % verwende KOMA-Report

%--------------------------------------
%   Packages
%--------------------------------------
\usepackage[utf8]{inputenc}   % Zeichenkodierung - !!!Achtung, alle Dateien auch im UTF8 speichern!!!
\usepackage[T1]{fontenc}      % westeuropäische Schriftzeichenkodierung
\usepackage[ngerman]{babel}   % Babel-System
\usepackage[left=4cm, right=3cm, top=2.5cm, bottom=2.5cm]{geometry} % Seitenränder
\usepackage{array}            % erweiterte Tabelleneigenschaften
\usepackage{graphicx}         % Grafiken
\usepackage{subfigure}        % Grafiken nebeneinander mit (a) und (b)

%---------------------------------------
%  Befehl für Schriftart Helvet / Arial
%---------------------------------------
%\renewcommand*{\familydefault}{\sfdefault}
%\usepackage[scaled]{helvet}

%%%%%%%%%%%%%%%%2 Alternative Schriftarten%%%%%%%%%%%%%%%%%%%%
\usepackage{lmodern}					% Schriftart
%\usepackage{palatino}          % Schriftart Palatino (besser am Bildschirm zu lesen)

\usepackage[onehalfspacing]{setspace} % 1,5 facher zeilen bastand aber nur im text nicht in Fußnoten oder verzeichnissen

\usepackage{amssymb}					% Mathesymbole
\usepackage{amsfonts}					% mathematische Schriftarten
\usepackage{amsmath}					% Mathepaket
\usepackage{cancel}						% Durchstreichungen wie beim kürzen
\usepackage{mathcomp}					% weitere Symbole
\usepackage{scrhack}          % verbessert einige Fremdklassen in Zusammenspiel mit KomaScript
\usepackage[babel,german=quotes]{csquotes}
\usepackage[ngerman]{translator}
\usepackage{epstopdf}
\usepackage{tikz}							% Für selbsterstellte Graphiken
\usepackage{booktabs}					% Für schönere Tabellen


%--------------------------------------
%   Metainformation
%--------------------------------------
\newcommand{\art}{Projektdokumentation}
\newcommand{\titel}{Scratch-Kurs}
\newcommand{\untertitel}{Praxiswerkstatt "AI4Kids"}
\newcommand{\autor}{Neo Gruber, Thomas Engstler, Christian Möser}
\newcommand{\registriernr}{}
\newcommand{\hochschule}{Fachhochschule Erfurt}
\newcommand{\fachgebiet}{Angewandte Informatik}
\newcommand{\erstgutachter}{Prof. Dr.-Ing. Oksana Arnold}
%\newcommand{\zweitgutachter}{Dr. XX YY}
%\newcommand{\datum}{\today} % \today gibt heutiges Datum aus, oder klassisch durch Abgabedatum zu ersetzen
\newcommand{\keywords}{aa,bb,cc,dd}
\newcommand{\ort}{Erfurt}

	
%--------------------------------------
%   PDF Lesezeichen und Hyperlinks
%--------------------------------------
\usepackage[
	pdfauthor={\autor},
	pdftitle={{\titel { - }\untertitel}},
	pdfsubject={{\titel { - }\untertitel}},
	pdfkeywords={\keywords},
	pdfpagelabels = {true},
	pdfstartview = {FitV},
	colorlinks = {true},
	linkcolor = {black},
	citecolor = {black},
	urlcolor = {blue},
	bookmarksopen = {true},
	bookmarksopenlevel = {3},
	bookmarksnumbered = {true},
	plainpages = {false},
	hypertexnames = {false}
]{hyperref}
%\usepackage[acronym,toc]{glossaries}

%--------------------------------------
%   Kopf- & Fußzeile 
%--------------------------------------
\usepackage[automark,plainheadsepline,autooneside]{scrpage2}
\pagestyle{scrheadings}
\setheadsepline{.4pt}                                   %Separate Linie im Kopf
\clearscrheadfoot                                       %Kopf und Fuzeile lschen
\ihead[\hochschule]{\hochschule}                        % im Kopf -> links
\ohead[\fachgebiet]{\fachgebiet}                        % im Kopf -> rechts
\cfoot[\pagemark]{\pagemark}                            % Seitenzahl
\renewcommand*{\headfont}{\upshape\sffamily\scriptsize} % Schrift Kopfzeile
\renewcommand*{\footfont}{\normalfont\sffamily\small}   % Schrift Fußzeile


%--------------------------------------
%   Quellcode-Listing Einstellungen
%   ftp://ftp.tu-chemnitz.de/pub/tex/macros/latex/contrib/listings/listings.pdf
%--------------------------------------

\usepackage{xcolor,listings}                %bindet das Paket Listings ein
\definecolor{comment}{rgb}{.15,.4,.15}     % hellgruen
\definecolor{keywd1}{rgb}{.15,.15,.6}      % dunkelblau
\definecolor{keywd2}{rgb}{.35,.5,.55}      % hellblau
\definecolor{string}{rgb}{.5,.15,.15}      % dunkelrot
\definecolor{gray}{rgb}{0.4,0.4,0.4}
\definecolor{darkblue}{rgb}{0.0,0.0,0.6}
\definecolor{cyan}{rgb}{0.0,0.6,0.6}

% Der lstset-Befehl ermöglicht haufenweise Einstellungen zur Formatierung


\lstset{language=C++,												%hier Sprache einstellen
basicstyle={\small} ,												%Schriftgröße
keywordstyle=\color{blue!80!black!100},			%Farbe der keywords
identifierstyle=,														%Bezeichnerstyle, hier leer
commentstyle=\color{green!50!black!100},		%Farbe der Kommentare
stringstyle=\ttfamily,											%Aussehen der Strings
breaklines=true,														%Automatische Zeilenumbrüche
numbers=left,																%Zeilennummerierung links
numberstyle=\small,													%Größe der Zeilennummerierung
frame=single,																%einfacher Rahmen
backgroundcolor=\color{blue!3},							%Hintergrundfarbe
caption={Code}, 														%Standardüberschrift
captionpos=t,																%Überschift oben (top)
% %UTF8 gebastle verdammtes tex grml
literate= %
{Ä}{{\"A}}1
{Ö}{{\"O}}1
{Ü}{{\"U}}1
{ß}{{\ss}}1
{ä}{{\"a}}1
{ö}{{\"o}}1
{ü}{{\"u}}1
{~}{{\textasciitilde}}1
}

%\lstdefinelanguage{XML}
%{
  %morestring=[b]",
  %morestring=[s]{>}{<},
  %morecomment=[s]{<?}{?>},
  %stringstyle=\color{black},
  %identifierstyle=\color{darkblue},
  %keywordstyle=\color{cyan},
  %morekeywords={xmlns,version,type}% list your attributes here
%}



%\renewcommand{\lstlistingname}{Quellcodeausschnitt}
%\renewcommand{\lstlistlistingname}{Quellcodeausschnittsverzeichnis}


%----------Worttrennung-----------
%---Latex trennt eigentlich recht gut, aber Fremdwörter o.ä. manchmal nicht, daher kann man Latex das Trennen einzelner Wörter beibringen, z.B.:
%\hyphenation{Ein-gangs-pro-zess}
%\hyphenation{Aus-gangs-pro-zess}
%\hyphenation{Web-client}
%\hyphenation{DMS-Web-client}
\makeindex
%--------------------------------------
%   Dokumentenbeginn
%--------------------------------------
\begin{document}

%--------------------------------------
%   Titelseite
%--------------------------------------
\begin{titlepage}
	\pdfbookmark[-1]{\titel}{Marke}					%Titel wird bei Lesezeichen angezeigt
	
	\titlehead{
		\begin{flushright}
			\includegraphics[width=0.45\textwidth]{images/Logo_Informatik.pdf}
		\end{flushright}
	}
	\subject{\art \\ \\ \large{\mdseries{\registriernr}}}	 % Art der Arbeit
	\title{\Huge\titel}                          					% Titel der Arbeit
	\subtitle{\Large\untertitel\\[5em]}										% ggf. Untertitel
	\author{\textbf{\autor}}                           		% Verfasser
	\date{Abgabedatum: 5. Mai 2025}                            % Datum
	\publishers{\erstgutachter \\ } 				% Betreuer
	\maketitle                                
\end{titlepage}

%--------------------------------------
%   Seitennummerierung
%--------------------------------------
\pagenumbering{Roman}

%--------------------------------------
%   Abstract und Kurzfassung
%--------------------------------------
\pdfbookmark[0]{Kurzfassung}{MKurzfassung}
\begin{abstract}
\section*{Kurzfassung}
\label{sec:Kurzfassung}
Diese projektbezogene Teamarbeit dokumentiert die Konzeption und Planung eines sechsteiligen Unterrichtsprojekts zur kindgerechten Einführung in die visuelle Programmiersprache Scratch im Rahmen der Praxiswerkstatt AI4Kids an der Fachhochschule Erfurt. Ziel des Projekts ist es, Kinder und Jugendliche im Alter von 8 bis 14 Jahren spielerisch an grundlegende Prinzipien des Programmierens heranzuführen und ihr Interesse an digitalen Technologien zu fördern. \\Die Unterrichtseinheiten umfassen jeweils 90 Minuten und wurden inhaltlich, methodisch und didaktisch detailliert ausgearbeitet. Dabei wurde besonderer Wert auf altersgerechte Ansprache, interaktive Elemente und projektbasiertes Lernen gelegt. Die Arbeit stellt die Unterrichtsplanung, eingesetzten Materialien sowie didaktischen Überlegungen vor und reflektiert erste Erfahrungen aus der Durchführung. Das Projekt leistet einen Beitrag zur Förderung digitaler Bildung und verdeutlicht die Potenziale praxisorientierter Lehrformate im schulnahen Kontext.

\end{abstract}

\pdfbookmark[0]{Abstract}{MAbstract}
\begin{abstract}
\section*{Abstract}
\label{sec:Abstract}
This project-related teamwork documents the conception and planning of a six-part teaching project for a child-friendly introduction to the visual programming language Scratch as part of the AI4Kids practical workshop at Erfurt University of Applied Sciences. The aim of the project is to introduce children and young people aged 8 to 14 to the basic principles of programming in a fun way and to encourage their interest in digital technologies.\\ The teaching units each last 90 minutes and have been developed in detail in terms of content, methodology and didactics. Particular emphasis was placed on an age-appropriate approach, interactive elements and project-based learning. The paper presents the lesson planning, materials used and didactic considerations and reflects on initial experiences from the implementation. The project contributes to the promotion of digital education and highlights the potential of practice-orientated teaching formats in a school context.
\end{abstract}


%--------------------------------------
%   Inhalts-, Abbildungs-, Tabellenverzeichnis
%--------------------------------------
\pdfbookmark[0]{\contentsname}{tocanc}
\tableofcontents  % Inhalt
\listoffigures    % Abbildungen
%\listoftables     % Tabellen
\lstlistoflistings %Listings
\cleardoublepage

%--------------------------------------
%   Zähler für römische Nummerierung
%--------------------------------------
\newcounter{exterior}
\setcounter{exterior}{\value{page}}

\pagenumbering{arabic}
%\include{chapter/glossarie}
%\makeglossaries



%--------------------------------------
%   Einfügen der Kapitel
%--------------------------------------
\setcounter{page}{1}
\pagenumbering{arabic}
\begin{onehalfspace}
\chapter{Einleitung}
\label{sec:Einleitung}
In einer zunehmend digitalisierten Welt wird die Fähigkeit, technologische Zusammenhänge zu verstehen und selbst aktiv mitzugestalten, immer wichtiger – auch für Kinder und Jugendliche. Im Rahmen der Praxiswerkstatt AI4Kids an der Fachhochschule Erfurt wurde ein Unterrichtsprojekt entwickelt, das Kindern im Alter von 8 bis 14 Jahren spielerisch erste Programmiererfahrungen mit der visuellen Programmiersprache Scratch vermittelt.
Ziel der vorliegenden Teamarbeit ist die detaillierte Planung und Dokumentation eines sechsteiligen Unterrichtsmoduls, das jungen Lernenden einen kreativen und altersgerechten Zugang zur digitalen Welt ermöglicht. Dabei werden nicht nur technische Grundlagen, sondern auch Problemlösungskompetenz, logisches Denken und Teamarbeit gefördert.
Die Arbeit gliedert sich wie folgt: Kapitel 2 stellt den theoretischen Hintergrund zu kindgerechter Programmierbildung und didaktischen Prinzipien dar. Kapitel 3 beschreibt die Zielgruppe und Rahmenbedingungen des Projekts. In Kapitel 4 erfolgt die konkrete Ausarbeitung der Unterrichtseinheiten, gefolgt von einer Reflexion und Diskussion erster Umsetzungserfahrungen in Kapitel 5. Kapitel 6 fasst die Ergebnisse zusammen und gibt einen Ausblick auf mögliche Weiterentwicklungen.

\chapter{Rahmenbedingungen}
\label{sec: Rahmenbedingungen}

\section{Projektkontext}
\label{sec:Projektkontext}

Das Projekt „Animationen mit Scratch“ wurde im Rahmen des Praxismoduls AI4Kids an der Fachhochschule Erfurt im Sommersemester 2025 durchgeführt. Der Unterricht fand in Haus 5, Etage 1, Raum 5 statt und war in zwei Teile untergliedert: Teil A und Teil B, die jeweils drei Unterrichtseinheiten à 90 Minuten umfassten. Die Termine für Teil A waren der 8. Mai, 15. Mai und 22. Mai 2025. Die Termine für Teil B wurden auf den 12. Juni, 19. Juni und 26. Juni 2025 festgelegt.

Die Zielgruppe bestand aus Kindern im Alter von 8 bis 14 Jahren, wobei die genaue Teilnehmerzahl und die Vorkenntnisse der Kinder vor dem Beginn des Kurses nicht bekannt waren.

Die Modulverantwortliche für das Praxismodul AI4Kids war Prof. Dr.-Ing. Oksana Arnold, die das Projekt fachlich begleitete und die Studierenden in ihrer Planung und Umsetzung unterstützte.
\section{Rolle der Studierenden}
\label{sec:Rolle der Studierenden}

Die Studierenden waren verantwortlich für die vollständige Planung und Durchführung des Unterrichts. Zu ihren Aufgaben gehörten die Erstellung der Unterrichtseinheiten, die Auswahl der didaktischen Methoden sowie die Gestaltung des Unterrichtsablaufs. Zudem lag es in ihrer Verantwortung, die Unterrichtsinhalte an die Bedürfnisse der Kinder anzupassen und die Teilnehmer aktiv in den Lernprozess einzubeziehen. Während des Projekts waren die Studierenden auch als Lehrende tätig, führten den Unterricht durch und sorgten für eine praxisorientierte und interaktive Vermittlung der Programmiersprache Scratch.
\section{Organisatorisches}
\label{sec:Organisatorisches}
Die Koordination des Projekts erfolgte durch regelmäßige Meetings und den Austausch von Projektunterlagen über GitHub. Die Kommunikation zwischen den Beteiligten wurde durch eine eigens eingerichtete WhatsApp-Gruppe organisiert. Die Vorgaben für das Projekt umfassten sechs Unterrichtseinheiten à 90 Minuten sowie die Verwendung von Scratch als Programmierwerkzeug. Die Ziele des Kurses und die Lernvorgaben waren vorab festgelegt, jedoch unterlag die didaktische Methodik und Unterrichtsgestaltung den Studierenden, die innerhalb dieser Rahmenbedingungen eigenständig arbeiteten.
\chapter{Didaktisch-methodische und technische Grundlagen}
\label{sec:Didaktisch-methodische und technische Grundlagen}

\section{Pädagogisch-didaktische Grundlagen}
\subsection{Lernen im Kindes- und Jugendalter}
Kinder und Jugendliche lernen anders als Erwachsene. Sie benötigen handlungsorientierte Zugänge, bei denen das aktive Tun im Vordergrund steht. Ihre Motivation speist sich besonders aus Neugier und spielerischen Elementen. Soziale Einbindung in Gruppen oder durch Austausch mit Gleichaltrigen spielt eine zentrale Rolle. Konkrete Erfahrungen und anschauliche Beispiele sind wichtig, da abstrakte Konzepte noch schwer zugänglich sind. Multisensorische Zugänge, die verschiedene Sinne ansprechen, unterstützen den Lernprozess besonders effektiv.
\subsection{Konstruktivistisches Lernen}
Der Konstruktivismus als Lerntheorie geht davon aus, dass Lernende ihr Wissen aktiv selbst aufbauen. Dabei setzen sich die Lernenden eigene Ziele und lernen am besten in authentischen Kontexten. Sozialer Austausch mit anderen ist ein wesentlicher Bestandteil dieses Lernprozesses. Fehler werden als natürliche Lernchance verstanden, nicht als Defizit. Die Lehrperson übernimmt hier die Rolle eines Begleiters, der Unterstützung bietet, statt Wissen frontal zu vermitteln. Diese Prinzipien lassen sich besonders gut durch Projektarbeit, entdeckendes Lernen oder die Arbeit mit Lernstationen umsetzen.
\subsection{Projektorientiertes und spielerisches Lernen}
Projektorientiertes Lernen ermöglicht Kindern und Jugendlichen, an realen Aufgabenstellungen zu arbeiten. Dieser Ansatz ist häufig fächerübergreifend angelegt und fördert Selbstständigkeit sowie Verantwortungsbewusstsein. Typischerweise durchlaufen die Lernenden dabei die Phasen Planung, Durchführung, Präsentation und Reflexion. Der große Vorteil dieser Methode liegt in der Steigerung der Motivation und der Nachhaltigkeit des Gelernten. Spielerisches Lernen nutzt natürliche Spielformen und Elemente der Gamification wie Punkte, Level oder Belohnungen. Diese Herangehensweise unterstützt besonders das kreative Denken und die Entwicklung von Problemlösefähigkeiten. Beispiele reichen von klassischen Lernspielen über Rollenspiele bis hin zu Quiz-Formaten.
\subsection{Differenzierung nach Altersgruppen}
Bei der Differenzierung nach Altersgruppen müssen Entwicklungsstände berücksichtigt werden. Kinder im Alter von 6-10 Jahren befinden sich in der konkret-operationalen Phase und haben eine begrenzte Aufmerksamkeitsspanne. Für sie eignen sich besonders Bewegungsspiele, einfache Experimente, Geschichten und Lernstationen mit klaren Anweisungen. Jugendliche zwischen 10-14 Jahren entwickeln bereits logisches Denken und zeigen eine stärkere Sozialorientierung. In dieser Phase sind Gruppenprojekte, Diskussionen und komplexere Experimente besonders geeignet. Ab etwa 14 Jahren wird abstraktes Denken möglich und die Identitätsentwicklung rückt in den Vordergrund. Hier können Projektarbeit, Debatten und selbstgesteuerte Lernformen eingesetzt werden, die stark mit der Lebenswelt der Jugendlichen verknüpft sind.
\section{Guter Unterricht – Theorien und Qualitätsmerkmale}
\subsection{Zehn Merkmale guten Unterrichts (Hilbert Meyer)}
Der renommierte Pädagoge Hilbert Meyer, dessen handlungsorientierter Ansatz unser Projekt maßgeblich beeinflusst, entwickelte in seinem Werk "Was ist guter Unterricht?" (2004) einen Kriterien-Mix, der sowohl Ergebnisse empirischer Studien als auch seine normative Sicht auf qualitativ hochwertigen Unterricht integriert. Diese normative Dimension betont die Frage nach dem "Sollen", insbesondere in Bezug auf Handlungsweisen, Werte und anzustrebende Ziele, was eine Reflexion im fachlichen Kontext unerlässlich macht. Die zehn Merkmale nach Meyer umfassen:
\begin{enumerate}
    \item \textbf{Klare Strukturierung des Unterrichts}
    \item \textbf{Hoher Anteil echter Lernzeit}
    \item \textbf{Lernförderliches Klima}
    \item \textbf{Inhaltliche Klarheit}
    \item \textbf{Sinnstiftendes Kommunizieren}
    \item \textbf{Methodenvielfalt}
    \item \textbf{Individuelles Fördern}
    \item \textbf{Intelligentes Üben}
    \item \textbf{Transparente Leistungserwartungen}
    \item \textbf{Vorbereitete Umgebung}
\end{enumerate}
\subsection{Kriterien für guten Unterricht (Hans Haenisch)}
Die von Hans Haenisch (2002) formulierten "Kriterien zu gutem Unterricht" fokussieren auf übergreifende Aspekte des Lehrens und Lernens und basieren auf empirischen Studien, die den Zusammenhang zwischen Unterrichtsprozessen und Schülerleistungen untersuchen. Zu diesen Kriterien gehören unter anderem:
\begin{enumerate}
    \item \textbf{Den Unterricht curricular klar ausrichten}
    \item \textbf{Orientierung geben}
    \item \textbf{Die aktive Beteiligung verstärken und Lerngelegenheiten bewusst gestalten}
    \item \textbf{Das bisherige Wissen berücksichtigen und entsprechend umstrukturieren}
    \item \textbf{Lernstrategien zeigen}
    \item \textbf{Gelegenheit bieten, das Gelernte zu üben und anzuwenden}
    \item \textbf{Aktivitäten und Lernfortschritte sorgfältig beobachten, kontrollieren, analysieren und Rückmeldungen geben}
    \item \textbf{Phasen kooperativen Lernens systematisch einbauen}
    \item \textbf{Für einen lernförderlichen Unterrichtskontext sorgen}
\end{enumerate}
\subsection{Unterrichtsqualität nach Andreas Helmke}
Andreas Helmke, ein Vertreter der empirischen Erziehungswissenschaft, präsentierte Merkmale der Unterrichtsqualität, die auf umfangreicher Forschung basieren. Dazu zählen:
\begin{itemize}
    \item \textbf{Strukturiertheit, Klarheit, Verständlichkeit}
    \item \textbf{Effiziente Klassenführung und Zeitnutzung}
    \item \textbf{Lernförderliches Unterrichtsklima}
    \item \textbf{Effiziente Klassenführung und Zeitnutzung}
    \item \textbf{Lernförderliches Unterrichtsklima}
    \item \textbf{Ziel-, Wirkungs- und Kompetenzorientierung}
    \item \textbf{Schülerorientierung, Unterstützung}
    \item \textbf{Angemessene Variation von Methoden und Sozialformen}
    \item \textbf{Aktivierung: Förderung aktiven, selbstständigen Lernens}
    \item \textbf{Konsolidierung, Sicherung, Intelligentes Üben}
    \item \textbf{Vielfältige Motivierung}
    \item \textbf{Passung: Umgang mit heterogenen Lernvoraussetzungen}
\end{itemize}
\subsection{Merkmale wirksamen Unterrichts (Marcus Pietsch)}
Marcus Pietsch betont, dass effektiver Unterricht weniger durch einzelne Merkmale als vielmehr durch das gelungene Zusammenspiel verschiedener Qualitätsaspekte ("Orchestrierung" oder "Choreografie" des Unterrichts) geprägt ist. Er verweist auf die Bedeutung tiefenstruktureller Merkmale für die Lernentwicklung der Schülerinnen und Schüler.
\subsection{Erkenntnisse der Schul- und Unterrichtsforschung (Hartmut Ditton)}
Die Forschung von Hartmut Ditton hebt zentrale Qualitätsdimensionen für den Unterricht hervor, die in unserem Projekt Beachtung finden:

Qualität (Quality): Struktur und Strukturiertheit, Klarheit, Methodenvariabilität, angemessenes Tempo, Mediennutzung, Übungsintensität, Stoffumfang, Leistungserwartungen, Motivierung, bedeutsame Inhalte, bekannte Ziele, Vermeidung von Leistungsangst, Interesse und Neugier wecken, Bekräftigung, positives Sozialklima.

Angemessenheit (Appropriateness): Schwierigkeitsgrad, Adaptivität, diagnostische Sensibilität, individuelle Unterstützung, Differenzierung, Förderungsorientierung.

Unterrichtszeit (Time): Verfügbare Zeit, Lerngelegenheiten, genutzte Lernzeit, Inhaltsorientierung, Klassenmanagement.
\section{Das Didaktische Sechseck (Hilbert Meyer) als Analyseinstrument}
Ein zentrales Analyseinstrument für die Konzeption und Reflexion unseres kompetenzorientierten Scratch-Unterrichts bildet das Didaktische Sechseck nach Hilbert Meyer. Dieses Modell umfasst sechs Kernelemente, die bei der Planung und Durchführung jeder Unterrichtseinheit berücksichtigt werden:
\begin{itemize}
    \item \textbf{Zielstruktur: Welche Kompetenzen und Lernziele sollen die Kinder erreichen?}
    \item \textbf{Inhaltsstruktur: Wie sind die Inhalte didaktisch aufbereitet und sequenziert?}
    \item \textbf{Prozessstruktur: Wie wird der Unterrichtsablauf gestaltet und welche Methoden kommen zum Einsatz?}
    \item \textbf{Handlungsstruktur: Welche Aktivitäten führen die Kinder durch und wie aktiv sind sie beteiligt?}
    \item \textbf{Sozialstruktur: Wie wird das soziale Miteinander im Lernprozess gefördert?}
    \item \textbf{Raumstruktur: Wie ist der Lernraum gestaltet, um das Lernen optimal zu unterstützen?}
\end{itemize}
Die bewusste Auseinandersetzung mit diesen sechs Dimensionen ermöglicht eine strukturierte Planung und Analyse des Unterrichtsgeschehens im Hinblick auf seine Qualität und Effektivität.
\section{Technische Grundlagen: Einführung in Scratch}
\subsection{Was ist Scratch?}
Scratch ist eine visuelle Programmiersprache, die speziell für Kinder und Einsteiger entwickelt wurde. Mit Scratch können Nutzer:innen interaktive Geschichten, Spiele und Animationen erstellen, ohne komplexen Code schreiben zu müssen. Die Plattform fördert spielerisches Lernen und ermöglicht es, grundlegende Programmierkonzepte auf kreative Weise zu verstehen.
Scratch wurde vom MIT Media Lab entwickelt und ist kostenlos nutzbar. Es eignet sich besonders für Kinder ab etwa 8 Jahren, aber auch ältere Lernende können damit die Grundlagen der Programmierung entdecken.
\subsection{Grundprinzipien der Programmierumgebung}
Scratch arbeitet mit visuellen Blöcken und einem Drag-and-Drop-System, das den Einstieg in die Programmierung vereinfacht:

Visuelle Blöcke:
Befehle werden als farbige Code-Blöcke dargestellt, die nach Kategorien (z.B. Bewegung, Ereignisse, Steuerung) sortiert sind.
Logische Abfolgen werden durch das Zusammenstecken der Blöcke gebildet – ähnlich wie Puzzle-Teile.
Drag-and-Drop-Oberfläche:
Nutzer:innen ziehen die Blöcke per Maus oder Touch in den Skriptbereich und fügen sie zusammen.
Dies vermeidet Tippfehler und macht Programmierung intuitiv zugänglich.

\subsection{Arbeiten mit Scratch aus Sicht der Kinder}
\subsection{Typische Projektbeispiele}
Mit Scratch lassen sich vielfältige Projekte umsetzen, z.B.:

Animationen:
Einfache Bewegungsabläufe (z.B. eine tanzende Figur) oder kurze Geschichten.
Spiele:
Klassiker wie „Pong“, „Maze“ (Labyrinth) oder eigene Jump-&-Run-Spiele.
Interaktive Anwendungen:
Digitale Grußkarten, Quizze oder Simulationen (z.B. ein virtuelles Haustier).

\subsection{Altersgerechte Funktionen und Anwendungsmöglichkeiten}
Scratch ist auf junge Nutzer:innen zugeschnitten:

Figuren (Sprites) & Hintergründe:
Große Bibliothek an vorgefertigten Charakteren und Szenen.
Eigene Zeichnungen oder Fotos können importiert werden.
Sounds & Musik:
Einfache Audiobearbeitung (z.B. Töne aufnehmen, Effekte hinzufügen).
Einfache Logik:
Grundlegende Programmierkonzepte wie Schleifen („wiederhole“), Bedingungen („falls… dann“) oder Variablen werden kindgerecht vermittelt.

\chapter{Kurskonzept und didaktische Umsetzung}
\label{sec:konzept}

\section{Unterrichtsplanung}
\subsection{Zielsetzungen}
\subsubsection{Fachliche Kompetenzen}
\subsubsection{Soziale und personale Kompetenzen}
\subsubsection{Differenzierungsstrategien}

\subsection{Didaktisch-methodisches Konzept}
\subsubsection{Leitideen und didaktische Prinzipien}
\subsubsection{Methodische Umsetzung im Unterricht}
\subsubsection{Medien und Materialien}

\subsection{Erfahrungen und Herausforderungen in der Umsetzung}

%\chapter{Grundlagen}
\label{sec:Grundlagen}
\section{Didaktisch-methodische Grundlagen}

\subsection{Lernen im Kindes- und Jugendalter }
\label{sec:Lernen im Kindes- und Jugendalter}
Kinder und Jugendliche lernen anders als Erwachsene. Sie benötigen handlungsorientierte Zugänge, bei denen das aktive Tun im Vordergrund steht. Ihre Motivation speist sich besonders aus Neugier und spielerischen Elementen. Soziale Einbindung in Gruppen oder durch Austausch mit Gleichaltrigen spielt eine zentrale Rolle. Konkrete Erfahrungen und anschauliche Beispiele sind wichtig, da abstrakte Konzepte noch schwer zugänglich sind. Multisensorische Zugänge, die verschiedene Sinne ansprechen, unterstützen den Lernprozess besonders effektiv.

\subsection{Konstruktivistisches Lernen}
\label{sec:Konstruktivistisches Lernen}
Der Konstruktivismus als Lerntheorie geht davon aus, dass Lernende ihr Wissen aktiv selbst aufbauen. Dabei setzen sich die Lernenden eigene Ziele und lernen am besten in authentischen Kontexten. Sozialer Austausch mit anderen ist ein wesentlicher Bestandteil dieses Lernprozesses. Fehler werden als natürliche Lernchance verstanden, nicht als Defizit. Die Lehrperson übernimmt hier die Rolle eines Begleiters, der Unterstützung bietet, statt Wissen frontal zu vermitteln. Diese Prinzipien lassen sich besonders gut durch Projektarbeit, entdeckendes Lernen oder die Arbeit mit Lernstationen umsetzen.
\subsection{Projektorientiertes und spielerisches Lernen}
\label{sec:Projektorientiertes und spielerisches Lernen}
Projektorientiertes Lernen ermöglicht Kindern und Jugendlichen, an realen Aufgabenstellungen zu arbeiten. Dieser Ansatz ist häufig fächerübergreifend angelegt und fördert Selbstständigkeit sowie Verantwortungsbewusstsein. Typischerweise durchlaufen die Lernenden dabei die Phasen Planung, Durchführung, Präsentation und Reflexion. Der große Vorteil dieser Methode liegt in der Steigerung der Motivation und der Nachhaltigkeit des Gelernten. Spielerisches Lernen nutzt natürliche Spielformen und Elemente der Gamification wie Punkte, Level oder Belohnungen. Diese Herangehensweise unterstützt besonders das kreative Denken und die Entwicklung von Problemlösefähigkeiten. Beispiele reichen von klassischen Lernspielen über Rollenspiele bis hin zu Quiz-Formaten.


\subsection{Differenzierung nach Altersgruppen}
\label{sec:Differenzierung nach Altersgruppen}
Bei der Differenzierung nach Altersgruppen müssen Entwicklungsstände berücksichtigt werden. Kinder im Alter von 6-10 Jahren befinden sich in der konkret-operationalen Phase und haben eine begrenzte Aufmerksamkeitsspanne. Für sie eignen sich besonders Bewegungsspiele, einfache Experimente, Geschichten und Lernstationen mit klaren Anweisungen. Jugendliche zwischen 10-14 Jahren entwickeln bereits logisches Denken und zeigen eine stärkere Sozialorientierung. In dieser Phase sind Gruppenprojekte, Diskussionen und komplexere Experimente besonders geeignet. Ab etwa 14 Jahren wird abstraktes Denken möglich und die Identitätsentwicklung rückt in den Vordergrund. Hier können Projektarbeit, Debatten und selbstgesteuerte Lernformen eingesetzt werden, die stark mit der Lebenswelt der Jugendlichen verknüpft sind.


\section{Inhaltlich-technische Grundlagen}
\label{sec:Inhaltlich-technische Grundlagen}
\subsection{Einführung in Scratch}
\label{Einführung in Scratch}
Scratch ist eine visuelle Programmiersprache, die speziell für Kinder und Einsteiger entwickelt wurde. Mit Scratch können Nutzer:innen interaktive Geschichten, Spiele und Animationen erstellen, ohne komplexen Code schreiben zu müssen. Die Plattform fördert spielerisches Lernen und ermöglicht es, grundlegende Programmierkonzepte auf kreative Weise zu verstehen.

Scratch wurde vom MIT Media Lab entwickelt und ist kostenlos nutzbar. Es eignet sich besonders für Kinder ab etwa 8 Jahren, aber auch ältere Lernende können damit die Grundlagen der Programmierung entdecken.
\subsection{Grundprinzip von Scratch}
\label{sec:Grundprinzip von Scratch}
Scratch arbeitet mit visuellen Blöcken und einem Drag-and-Drop-System, das den Einstieg in die Programmierung vereinfacht:

Visuelle Blöcke:
Befehle werden als farbige Code-Blöcke dargestellt, die nach Kategorien (z.B. Bewegung, Ereignisse, Steuerung) sortiert sind.
Logische Abfolgen werden durch das Zusammenstecken der Blöcke gebildet – ähnlich wie Puzzle-Teile.
Drag-and-Drop-Oberfläche:
Nutzer:innen ziehen die Blöcke per Maus oder Touch in den Skriptbereich und fügen sie zusammen.
Dies vermeidet Tippfehler und macht Programmierung intuitiv zugänglich.

\subsection{Wie Kinder damit arbeiten können}
\label{sec:Wie Kinder damit arbeiten können}

Scratch fördert kreatives und spielerisches Lernen:

Spielerischer Ansatz:
Kinder experimentieren mit Figuren („Sprites“), Hintergründen und Sounds, um eigene Projekte zu gestalten.
Durch Ausprobieren lernen sie Ursache-Wirkung-Prinzipien (z.B. „Wenn Taste gedrückt wird, bewegt sich die Figur“).
Kreativität und Problemlösen:
Eigenes Gestalten steht im Vordergrund – es gibt kein „richtig“ oder „falsch“.
Kinder entwickeln Lösungen, indem sie Blöcke kombinieren (z.B. für eine Figur, die Hindernissen ausweicht).
\subsection{Typische Projekte}
\label{sec:Typische Projekte}
Mit Scratch lassen sich vielfältige Projekte umsetzen, z.B.:

Animationen:
Einfache Bewegungsabläufe (z.B. eine tanzende Figur) oder kurze Geschichten.
Spiele:
Klassiker wie „Pong“, „Maze“ (Labyrinth) oder eigene Jump-&-Run-Spiele.
Interaktive Anwendungen:
Digitale Grußkarten, Quizze oder Simulationen (z.B. ein virtuelles Haustier).

\subsection{Altersgerechte Funktionen}
\label{sec:Altersgerechte Funktionen}
Scratch ist auf junge Nutzer:innen zugeschnitten:

Figuren (Sprites) & Hintergründe:
Große Bibliothek an vorgefertigten Charakteren und Szenen.
Eigene Zeichnungen oder Fotos können importiert werden.
Sounds & Musik:
Einfache Audiobearbeitung (z.B. Töne aufnehmen, Effekte hinzufügen).
Einfache Logik:
Grundlegende Programmierkonzepte wie Schleifen („wiederhole“), Bedingungen („falls… dann“) oder Variablen werden kindgerecht vermittelt.



%\chapter{Unterrichtsplanung}
\label{Unterrichtsplanung}
\section{Ziele}
\label{sec:Ziele}
\subsection{Fachkompetenzen}
\label{Fachkompetenzen}
Die Teilnehmenden lernen, wie sie mit visuellen Code-Blöcken eigene interaktive Projekte wie Spiele, Animationen oder Geschichten erstellen können. Dabei entwickeln sie ein Verständnis für grundlegende Programmierkonzepte wie Schleifen, Bedingungen und Variablen.
\subsection{Personale und soziale Kompetenzen}
\label{sec:Personale und soziale Kompetenzen}
Neben den fachlichen Kompetenzen fördert der Kurs auch persönliche und soziale Fähigkeiten. Die Kinder trainieren ihre Problemlösefähigkeit, indem sie Fehler in ihren Programmen suchen und beheben. Bei Partner- oder Gruppenaufgaben üben sie Teamarbeit und Kommunikation. Durch das Präsentieren ihrer Projekte vor der Gruppe gewinnen sie an Selbstvertrauen und lernen, ihre Ideen verständlich zu erklären.
\subsection{Differenzierung}
\label{Differenzierung}
Der Kurs ist so gestaltet, dass er unterschiedliche Altersstufen und Vorkenntnisse berücksichtigt. Jüngere Kinder (8-10 Jahre) arbeiten mit einfacheren, kleinschrittigen Aufgaben, während ältere Teilnehmende (11-14 Jahre) komplexere Herausforderungen mit erweiterten Funktionen meistern können.
\section{Konzept}
\label{sec:Konzept}
\subsection{Didaktische Leitideen}
\label{sec:Didaktische Leitideen}
Der Kurs basiert auf einem konstruktivistischen Lernansatz, bei dem die Kinder durch aktives Experimentieren und Ausprobieren selbstständig Programmierkonzepte entdecken. Die Lernenden bauen ihr Wissen eigenständig auf, indem sie mit den Scratch-Blöcken arbeiten und eigene Projekte entwickeln. Fehler werden als natürlicher Teil des Lernprozesses betrachtet und führen zu wertvollen Aha-Momenten.

Ein weiterer zentraler Grundsatz ist die Projektorientierung. Jede Einheit mündet in einem konkreten, greifbaren Ergebnis, das die Kinder stolz präsentieren können. Dieser Ansatz fördert die Motivation und zeigt den praktischen Nutzen des Gelernten. Durch die Arbeit an eigenen Projekten entwickeln die Teilnehmenden nicht nur technische Fähigkeiten, sondern auch Kreativität und Problemlösekompetenz.

Der Kurs nutzt zudem spielerische Elemente (Gamification), um die natürliche Neugier und den Entdeckerdrang der Kinder anzusprechen. Kleine Challenges, Belohnungssysteme und kreative Freiheiten sorgen für eine lockere, motivierende Atmosphäre.

\subsection{Methodische Umsetzung}
\label{sec:Methodische Umsetzung}
Die Inputphasen sind bewusst kurz gehalten und dienen lediglich als Impulsgeber. Die Lehrkraft demonstriert beispielsweise neue Blöcke oder Funktionen an einem einfachen Beispiel, bevor die Kinder selbst aktiv werden.

Der Großteil der Kurszeit besteht aus Praxisphasen, in denen die Teilnehmenden eigenständig oder in Partnerarbeit an ihren Projekten arbeiten. Hier kommen differenzierte Materialien wie Schritt-für-Schritt-Anleitungen für Einsteiger oder offene Challenges für Fortgeschrittene zum Einsatz.

Soziale Lernformen wie Partnerarbeit, Gruppenreflexionen und Präsentationen fördern den Austausch unter den Kindern. Regelmäßige Feedbackrunden, in denen die Teilnehmenden ihre Projekte vorstellen und konstruktive Rückmeldung geben, stärken zudem die Kommunikationsfähigkeit.

Zur Differenzierung werden Aufgaben in unterschiedlichen Schwierigkeitsgraden angeboten. Jüngere oder unerfahrenere Kinder erhalten kleinschrittigere Anleitungen, während ältere oder schnellere Teilnehmende erweiterte Herausforderungen (z. B. komplexere Programmiertechniken) bearbeiten können.

Am Ende jeder Einheit steht eine Reflexionsphase, in der die Kinder ihr Vorgehen, ihre Lösungswege und ihre Lernerfolge gemeinsam besprechen. Diese Methode festigt das Gelernte und fördert die Fähigkeit, über den eigenen Lernprozess nachzudenken.

Durch diese Kombination aus klar strukturierten Inputs, kreativen Praxisphasen und kooperativen Lernformen entsteht ein abwechslungsreicher Kurs, der sowohl fachliche als auch soziale Kompetenzen fördert.
\chapter{Reflexion und Ausblick}
\label{Reflexion und Ausblick}
\include{chapter/fazit}
\end{onehalfspace}



%--------------------------------------
%   Seitennummerierung - Römisch
%--------------------------------------
\pagenumbering{Roman}
\setcounter{page}{\value{exterior}}

%--------------------------------------
%   Literaturverzeichnis mit BiBTeX
%--------------------------------------
\nocite{*} % baue alle Eintrge aus bib/literatur ein
\bibliographystyle{alphadin}

\bibliography{bib/literatur}
\clearpage
%\glsaddall
%\printglossaries
%--------------------------------------
%   Anhang
%--------------------------------------
\addcontentsline{toc}{chapter}{\appendixname}
\begin{appendix} 
	\chapter{Skripte}
%	\lstinputlisting[caption={Converterprogramm von alten zu neuen Konfigurationsdateien},label=app:conv]{Quellen/env2imp.java}
%	\lstinputlisting[language=ANT,caption={Angepastes Ant-Buildskript},label=app:antinstall]{Quellen/build.xml}
%	\lstinputlisting[caption={Batchdatei zum Automatischen Verteilen},language=command.com,label=app:batchjenkins]{Quellen/depoly_DMSCore.bat}
	\chapter{Konfigurationen}
%	\lstinputlisting[caption={Alte Konfiguration},label=konf:oldkonf]{Quellen/Environment.properties}
%	\lstinputlisting[caption={Importdatei mit den Werten der alten Konfigurationsdatei},language=XML,label=konf:xmlfuerdms]{Quellen/out.xml}



\end{appendix}
\clearpage

%--------------------------------------
%   Selbststndigkeitserklrung
%--------------------------------------
\addchap{Selbstständigkeitserklärung}
Ich, \autor, versichere hiermit, dass ich die vorliegende \art  { }mit dem Thema

\begin{quote}
\textit{\titel} \textit{\untertitel}
\end{quote}

selbstständig und nur unter Verwendung der angegebenen Quellen und Hilfsmittel angefertigt habe.

\begin{flushright}
\ort, \datum 
\end{flushright}
\small{\autor} 

%--------------------------------------
%   Dokumentenende
%--------------------------------------
\end{document}


