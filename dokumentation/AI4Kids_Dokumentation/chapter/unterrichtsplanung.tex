\chapter{Unterrichtsplanung}
\label{Unterrichtsplanung}
\section{Ziele}
\label{sec:Ziele}
\subsection{Fachkompetenzen}
\label{Fachkompetenzen}
Die Teilnehmenden lernen, wie sie mit visuellen Code-Blöcken eigene interaktive Projekte wie Spiele, Animationen oder Geschichten erstellen können. Dabei entwickeln sie ein Verständnis für grundlegende Programmierkonzepte wie Schleifen, Bedingungen und Variablen.
\subsection{Personale und soziale Kompetenzen}
\label{sec:Personale und soziale Kompetenzen}
Neben den fachlichen Kompetenzen fördert der Kurs auch persönliche und soziale Fähigkeiten. Die Kinder trainieren ihre Problemlösefähigkeit, indem sie Fehler in ihren Programmen suchen und beheben. Bei Partner- oder Gruppenaufgaben üben sie Teamarbeit und Kommunikation. Durch das Präsentieren ihrer Projekte vor der Gruppe gewinnen sie an Selbstvertrauen und lernen, ihre Ideen verständlich zu erklären.
\subsection{Differenzierung}
\label{Differenzierung}
Der Kurs ist so gestaltet, dass er unterschiedliche Altersstufen und Vorkenntnisse berücksichtigt. Jüngere Kinder (8-10 Jahre) arbeiten mit einfacheren, kleinschrittigen Aufgaben, während ältere Teilnehmende (11-14 Jahre) komplexere Herausforderungen mit erweiterten Funktionen meistern können.
\section{Konzept}
\label{sec:Konzept}
\subsection{Didaktische Leitideen}
\label{sec:Didaktische Leitideen}
Der Kurs basiert auf einem konstruktivistischen Lernansatz, bei dem die Kinder durch aktives Experimentieren und Ausprobieren selbstständig Programmierkonzepte entdecken. Die Lernenden bauen ihr Wissen eigenständig auf, indem sie mit den Scratch-Blöcken arbeiten und eigene Projekte entwickeln. Fehler werden als natürlicher Teil des Lernprozesses betrachtet und führen zu wertvollen Aha-Momenten.

Ein weiterer zentraler Grundsatz ist die Projektorientierung. Jede Einheit mündet in einem konkreten, greifbaren Ergebnis, das die Kinder stolz präsentieren können. Dieser Ansatz fördert die Motivation und zeigt den praktischen Nutzen des Gelernten. Durch die Arbeit an eigenen Projekten entwickeln die Teilnehmenden nicht nur technische Fähigkeiten, sondern auch Kreativität und Problemlösekompetenz.

Der Kurs nutzt zudem spielerische Elemente (Gamification), um die natürliche Neugier und den Entdeckerdrang der Kinder anzusprechen. Kleine Challenges, Belohnungssysteme und kreative Freiheiten sorgen für eine lockere, motivierende Atmosphäre.

\subsection{Methodische Umsetzung}
\label{sec:Methodische Umsetzung}
Die Inputphasen sind bewusst kurz gehalten und dienen lediglich als Impulsgeber. Die Lehrkraft demonstriert beispielsweise neue Blöcke oder Funktionen an einem einfachen Beispiel, bevor die Kinder selbst aktiv werden.

Der Großteil der Kurszeit besteht aus Praxisphasen, in denen die Teilnehmenden eigenständig oder in Partnerarbeit an ihren Projekten arbeiten. Hier kommen differenzierte Materialien wie Schritt-für-Schritt-Anleitungen für Einsteiger oder offene Challenges für Fortgeschrittene zum Einsatz.

Soziale Lernformen wie Partnerarbeit, Gruppenreflexionen und Präsentationen fördern den Austausch unter den Kindern. Regelmäßige Feedbackrunden, in denen die Teilnehmenden ihre Projekte vorstellen und konstruktive Rückmeldung geben, stärken zudem die Kommunikationsfähigkeit.

Zur Differenzierung werden Aufgaben in unterschiedlichen Schwierigkeitsgraden angeboten. Jüngere oder unerfahrenere Kinder erhalten kleinschrittigere Anleitungen, während ältere oder schnellere Teilnehmende erweiterte Herausforderungen (z. B. komplexere Programmiertechniken) bearbeiten können.

Am Ende jeder Einheit steht eine Reflexionsphase, in der die Kinder ihr Vorgehen, ihre Lösungswege und ihre Lernerfolge gemeinsam besprechen. Diese Methode festigt das Gelernte und fördert die Fähigkeit, über den eigenen Lernprozess nachzudenken.

Durch diese Kombination aus klar strukturierten Inputs, kreativen Praxisphasen und kooperativen Lernformen entsteht ein abwechslungsreicher Kurs, der sowohl fachliche als auch soziale Kompetenzen fördert.