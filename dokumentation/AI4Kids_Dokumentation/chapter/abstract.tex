\pdfbookmark[0]{Kurzfassung}{MKurzfassung}
\begin{abstract}
\section*{Kurzfassung}
\label{sec:Kurzfassung}
Diese projektbezogene Teamarbeit dokumentiert die Konzeption und Planung eines sechsteiligen Unterrichtsprojekts zur kindgerechten Einführung in die visuelle Programmiersprache Scratch im Rahmen der Praxiswerkstatt AI4Kids an der Fachhochschule Erfurt. Ziel des Projekts ist es, Kinder und Jugendliche im Alter von 8 bis 14 Jahren spielerisch an grundlegende Prinzipien des Programmierens heranzuführen und ihr Interesse an digitalen Technologien zu fördern. \\Die Unterrichtseinheiten umfassen jeweils 90 Minuten und wurden inhaltlich, methodisch und didaktisch detailliert ausgearbeitet. Dabei wurde besonderer Wert auf altersgerechte Ansprache, interaktive Elemente und projektbasiertes Lernen gelegt. Die Arbeit stellt die Unterrichtsplanung, eingesetzten Materialien sowie didaktischen Überlegungen vor und reflektiert erste Erfahrungen aus der Durchführung. Das Projekt leistet einen Beitrag zur Förderung digitaler Bildung und verdeutlicht die Potenziale praxisorientierter Lehrformate im schulnahen Kontext.

\end{abstract}

\pdfbookmark[0]{Abstract}{MAbstract}
\begin{abstract}
\section*{Abstract}
\label{sec:Abstract}
This project-related teamwork documents the conception and planning of a six-part teaching project for a child-friendly introduction to the visual programming language Scratch as part of the AI4Kids practical workshop at Erfurt University of Applied Sciences. The aim of the project is to introduce children and young people aged 8 to 14 to the basic principles of programming in a fun way and to encourage their interest in digital technologies.\\ The teaching units each last 90 minutes and have been developed in detail in terms of content, methodology and didactics. Particular emphasis was placed on an age-appropriate approach, interactive elements and project-based learning. The paper presents the lesson planning, materials used and didactic considerations and reflects on initial experiences from the implementation. The project contributes to the promotion of digital education and highlights the potential of practice-orientated teaching formats in a school context.
\end{abstract}
