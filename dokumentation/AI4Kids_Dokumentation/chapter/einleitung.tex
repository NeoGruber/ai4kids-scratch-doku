\chapter{Einleitung}
\label{sec:Einleitung}
In einer zunehmend digitalisierten Welt wird die Fähigkeit, technologische Zusammenhänge zu verstehen und selbst aktiv mitzugestalten, immer wichtiger – auch für Kinder und Jugendliche. Im Rahmen der Praxiswerkstatt AI4Kids an der Fachhochschule Erfurt wurde ein Unterrichtsprojekt entwickelt, das Kindern im Alter von 8 bis 14 Jahren spielerisch erste Programmiererfahrungen mit der visuellen Programmiersprache Scratch vermittelt.
Ziel der vorliegenden Teamarbeit ist die detaillierte Planung und Dokumentation eines sechsteiligen Unterrichtsmoduls, das jungen Lernenden einen kreativen und altersgerechten Zugang zur digitalen Welt ermöglicht. Dabei werden nicht nur technische Grundlagen, sondern auch Problemlösungskompetenz, logisches Denken und Teamarbeit gefördert.
Die Arbeit gliedert sich wie folgt: Kapitel 2 stellt den theoretischen Hintergrund zu kindgerechter Programmierbildung und didaktischen Prinzipien dar. Kapitel 3 beschreibt die Zielgruppe und Rahmenbedingungen des Projekts. In Kapitel 4 erfolgt die konkrete Ausarbeitung der Unterrichtseinheiten, gefolgt von einer Reflexion und Diskussion erster Umsetzungserfahrungen in Kapitel 5. Kapitel 6 fasst die Ergebnisse zusammen und gibt einen Ausblick auf mögliche Weiterentwicklungen.
