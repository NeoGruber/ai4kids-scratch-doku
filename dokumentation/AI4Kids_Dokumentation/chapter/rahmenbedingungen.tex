\chapter{Rahmenbedingungen}
\label{sec: Rahmenbedingungen}

\section{Projektkontext}
\label{sec:Projektkontext}

Das Projekt „Animationen mit Scratch“ wurde im Rahmen des Praxismoduls AI4Kids an der Fachhochschule Erfurt im Sommersemester 2025 durchgeführt. Der Unterricht fand in Haus 5, Etage 1, Raum 5 statt und war in zwei Teile untergliedert: Teil A und Teil B, die jeweils drei Unterrichtseinheiten à 90 Minuten umfassten. Die Termine für Teil A waren der 8. Mai, 15. Mai und 22. Mai 2025. Die Termine für Teil B wurden auf den 12. Juni, 19. Juni und 26. Juni 2025 festgelegt.

Die Zielgruppe bestand aus Kindern im Alter von 8 bis 14 Jahren, wobei die genaue Teilnehmerzahl und die Vorkenntnisse der Kinder vor dem Beginn des Kurses nicht bekannt waren.

Die Modulverantwortliche für das Praxismodul AI4Kids war Prof. Dr.-Ing. Oksana Arnold, die das Projekt fachlich begleitete und die Studierenden in ihrer Planung und Umsetzung unterstützte.
\section{Rolle der Studierenden}
\label{sec:Rolle der Studierenden}

Die Studierenden waren verantwortlich für die vollständige Planung und Durchführung des Unterrichts. Zu ihren Aufgaben gehörten die Erstellung der Unterrichtseinheiten, die Auswahl der didaktischen Methoden sowie die Gestaltung des Unterrichtsablaufs. Zudem lag es in ihrer Verantwortung, die Unterrichtsinhalte an die Bedürfnisse der Kinder anzupassen und die Teilnehmer aktiv in den Lernprozess einzubeziehen. Während des Projekts waren die Studierenden auch als Lehrende tätig, führten den Unterricht durch und sorgten für eine praxisorientierte und interaktive Vermittlung der Programmiersprache Scratch.
\section{Organisatorisches}
\label{sec:Organisatorisches}
Die Koordination des Projekts erfolgte durch regelmäßige Meetings und den Austausch von Projektunterlagen über GitHub. Die Kommunikation zwischen den Beteiligten wurde durch eine eigens eingerichtete WhatsApp-Gruppe organisiert. Die Vorgaben für das Projekt umfassten sechs Unterrichtseinheiten à 90 Minuten sowie die Verwendung von Scratch als Programmierwerkzeug. Die Ziele des Kurses und die Lernvorgaben waren vorab festgelegt, jedoch unterlag die didaktische Methodik und Unterrichtsgestaltung den Studierenden, die innerhalb dieser Rahmenbedingungen eigenständig arbeiteten.