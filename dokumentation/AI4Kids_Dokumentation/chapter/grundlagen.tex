\chapter{Grundlagen}
\label{sec:Grundlagen}
\section{Didaktisch-methodische Grundlagen}

\subsection{Lernen im Kindes- und Jugendalter }
\label{sec:Lernen im Kindes- und Jugendalter}
Kinder und Jugendliche lernen anders als Erwachsene. Sie benötigen handlungsorientierte Zugänge, bei denen das aktive Tun im Vordergrund steht. Ihre Motivation speist sich besonders aus Neugier und spielerischen Elementen. Soziale Einbindung in Gruppen oder durch Austausch mit Gleichaltrigen spielt eine zentrale Rolle. Konkrete Erfahrungen und anschauliche Beispiele sind wichtig, da abstrakte Konzepte noch schwer zugänglich sind. Multisensorische Zugänge, die verschiedene Sinne ansprechen, unterstützen den Lernprozess besonders effektiv.

\subsection{Konstruktivistisches Lernen}
\label{sec:Konstruktivistisches Lernen}
Der Konstruktivismus als Lerntheorie geht davon aus, dass Lernende ihr Wissen aktiv selbst aufbauen. Dabei setzen sich die Lernenden eigene Ziele und lernen am besten in authentischen Kontexten. Sozialer Austausch mit anderen ist ein wesentlicher Bestandteil dieses Lernprozesses. Fehler werden als natürliche Lernchance verstanden, nicht als Defizit. Die Lehrperson übernimmt hier die Rolle eines Begleiters, der Unterstützung bietet, statt Wissen frontal zu vermitteln. Diese Prinzipien lassen sich besonders gut durch Projektarbeit, entdeckendes Lernen oder die Arbeit mit Lernstationen umsetzen.
\subsection{Projektorientiertes und spielerisches Lernen}
\label{sec:Projektorientiertes und spielerisches Lernen}
Projektorientiertes Lernen ermöglicht Kindern und Jugendlichen, an realen Aufgabenstellungen zu arbeiten. Dieser Ansatz ist häufig fächerübergreifend angelegt und fördert Selbstständigkeit sowie Verantwortungsbewusstsein. Typischerweise durchlaufen die Lernenden dabei die Phasen Planung, Durchführung, Präsentation und Reflexion. Der große Vorteil dieser Methode liegt in der Steigerung der Motivation und der Nachhaltigkeit des Gelernten. Spielerisches Lernen nutzt natürliche Spielformen und Elemente der Gamification wie Punkte, Level oder Belohnungen. Diese Herangehensweise unterstützt besonders das kreative Denken und die Entwicklung von Problemlösefähigkeiten. Beispiele reichen von klassischen Lernspielen über Rollenspiele bis hin zu Quiz-Formaten.


\subsection{Differenzierung nach Altersgruppen}
\label{sec:Differenzierung nach Altersgruppen}
Bei der Differenzierung nach Altersgruppen müssen Entwicklungsstände berücksichtigt werden. Kinder im Alter von 6-10 Jahren befinden sich in der konkret-operationalen Phase und haben eine begrenzte Aufmerksamkeitsspanne. Für sie eignen sich besonders Bewegungsspiele, einfache Experimente, Geschichten und Lernstationen mit klaren Anweisungen. Jugendliche zwischen 10-14 Jahren entwickeln bereits logisches Denken und zeigen eine stärkere Sozialorientierung. In dieser Phase sind Gruppenprojekte, Diskussionen und komplexere Experimente besonders geeignet. Ab etwa 14 Jahren wird abstraktes Denken möglich und die Identitätsentwicklung rückt in den Vordergrund. Hier können Projektarbeit, Debatten und selbstgesteuerte Lernformen eingesetzt werden, die stark mit der Lebenswelt der Jugendlichen verknüpft sind.


\section{Inhaltlich-technische Grundlagen}
\label{sec:Inhaltlich-technische Grundlagen}
\subsection{Einführung in Scratch}
\label{Einführung in Scratch}
Scratch ist eine visuelle Programmiersprache, die speziell für Kinder und Einsteiger entwickelt wurde. Mit Scratch können Nutzer:innen interaktive Geschichten, Spiele und Animationen erstellen, ohne komplexen Code schreiben zu müssen. Die Plattform fördert spielerisches Lernen und ermöglicht es, grundlegende Programmierkonzepte auf kreative Weise zu verstehen.

Scratch wurde vom MIT Media Lab entwickelt und ist kostenlos nutzbar. Es eignet sich besonders für Kinder ab etwa 8 Jahren, aber auch ältere Lernende können damit die Grundlagen der Programmierung entdecken.
\subsection{Grundprinzip von Scratch}
\label{sec:Grundprinzip von Scratch}
Scratch arbeitet mit visuellen Blöcken und einem Drag-and-Drop-System, das den Einstieg in die Programmierung vereinfacht:

Visuelle Blöcke:
Befehle werden als farbige Code-Blöcke dargestellt, die nach Kategorien (z.B. Bewegung, Ereignisse, Steuerung) sortiert sind.
Logische Abfolgen werden durch das Zusammenstecken der Blöcke gebildet – ähnlich wie Puzzle-Teile.
Drag-and-Drop-Oberfläche:
Nutzer:innen ziehen die Blöcke per Maus oder Touch in den Skriptbereich und fügen sie zusammen.
Dies vermeidet Tippfehler und macht Programmierung intuitiv zugänglich.

\subsection{Wie Kinder damit arbeiten können}
\label{sec:Wie Kinder damit arbeiten können}

Scratch fördert kreatives und spielerisches Lernen:

Spielerischer Ansatz:
Kinder experimentieren mit Figuren („Sprites“), Hintergründen und Sounds, um eigene Projekte zu gestalten.
Durch Ausprobieren lernen sie Ursache-Wirkung-Prinzipien (z.B. „Wenn Taste gedrückt wird, bewegt sich die Figur“).
Kreativität und Problemlösen:
Eigenes Gestalten steht im Vordergrund – es gibt kein „richtig“ oder „falsch“.
Kinder entwickeln Lösungen, indem sie Blöcke kombinieren (z.B. für eine Figur, die Hindernissen ausweicht).
\subsection{Typische Projekte}
\label{sec:Typische Projekte}
Mit Scratch lassen sich vielfältige Projekte umsetzen, z.B.:

Animationen:
Einfache Bewegungsabläufe (z.B. eine tanzende Figur) oder kurze Geschichten.
Spiele:
Klassiker wie „Pong“, „Maze“ (Labyrinth) oder eigene Jump-&-Run-Spiele.
Interaktive Anwendungen:
Digitale Grußkarten, Quizze oder Simulationen (z.B. ein virtuelles Haustier).

\subsection{Altersgerechte Funktionen}
\label{sec:Altersgerechte Funktionen}
Scratch ist auf junge Nutzer:innen zugeschnitten:

Figuren (Sprites) & Hintergründe:
Große Bibliothek an vorgefertigten Charakteren und Szenen.
Eigene Zeichnungen oder Fotos können importiert werden.
Sounds & Musik:
Einfache Audiobearbeitung (z.B. Töne aufnehmen, Effekte hinzufügen).
Einfache Logik:
Grundlegende Programmierkonzepte wie Schleifen („wiederhole“), Bedingungen („falls… dann“) oder Variablen werden kindgerecht vermittelt.


