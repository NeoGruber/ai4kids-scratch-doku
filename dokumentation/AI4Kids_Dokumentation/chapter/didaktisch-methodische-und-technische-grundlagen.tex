\chapter{Didaktisch-methodische und technische Grundlagen}
\label{sec:Didaktisch-methodische und technische Grundlagen}

\section{Pädagogisch-didaktische Grundlagen}
\subsection{Lernen im Kindes- und Jugendalter}
Kinder und Jugendliche lernen anders als Erwachsene. Sie benötigen handlungsorientierte Zugänge, bei denen das aktive Tun im Vordergrund steht. Ihre Motivation speist sich besonders aus Neugier und spielerischen Elementen. Soziale Einbindung in Gruppen oder durch Austausch mit Gleichaltrigen spielt eine zentrale Rolle. Konkrete Erfahrungen und anschauliche Beispiele sind wichtig, da abstrakte Konzepte noch schwer zugänglich sind. Multisensorische Zugänge, die verschiedene Sinne ansprechen, unterstützen den Lernprozess besonders effektiv.
\subsection{Konstruktivistisches Lernen}
Der Konstruktivismus als Lerntheorie geht davon aus, dass Lernende ihr Wissen aktiv selbst aufbauen. Dabei setzen sich die Lernenden eigene Ziele und lernen am besten in authentischen Kontexten. Sozialer Austausch mit anderen ist ein wesentlicher Bestandteil dieses Lernprozesses. Fehler werden als natürliche Lernchance verstanden, nicht als Defizit. Die Lehrperson übernimmt hier die Rolle eines Begleiters, der Unterstützung bietet, statt Wissen frontal zu vermitteln. Diese Prinzipien lassen sich besonders gut durch Projektarbeit, entdeckendes Lernen oder die Arbeit mit Lernstationen umsetzen.
\subsection{Projektorientiertes und spielerisches Lernen}
Projektorientiertes Lernen ermöglicht Kindern und Jugendlichen, an realen Aufgabenstellungen zu arbeiten. Dieser Ansatz ist häufig fächerübergreifend angelegt und fördert Selbstständigkeit sowie Verantwortungsbewusstsein. Typischerweise durchlaufen die Lernenden dabei die Phasen Planung, Durchführung, Präsentation und Reflexion. Der große Vorteil dieser Methode liegt in der Steigerung der Motivation und der Nachhaltigkeit des Gelernten. Spielerisches Lernen nutzt natürliche Spielformen und Elemente der Gamification wie Punkte, Level oder Belohnungen. Diese Herangehensweise unterstützt besonders das kreative Denken und die Entwicklung von Problemlösefähigkeiten. Beispiele reichen von klassischen Lernspielen über Rollenspiele bis hin zu Quiz-Formaten.
\subsection{Differenzierung nach Altersgruppen}
Bei der Differenzierung nach Altersgruppen müssen Entwicklungsstände berücksichtigt werden. Kinder im Alter von 6-10 Jahren befinden sich in der konkret-operationalen Phase und haben eine begrenzte Aufmerksamkeitsspanne. Für sie eignen sich besonders Bewegungsspiele, einfache Experimente, Geschichten und Lernstationen mit klaren Anweisungen. Jugendliche zwischen 10-14 Jahren entwickeln bereits logisches Denken und zeigen eine stärkere Sozialorientierung. In dieser Phase sind Gruppenprojekte, Diskussionen und komplexere Experimente besonders geeignet. Ab etwa 14 Jahren wird abstraktes Denken möglich und die Identitätsentwicklung rückt in den Vordergrund. Hier können Projektarbeit, Debatten und selbstgesteuerte Lernformen eingesetzt werden, die stark mit der Lebenswelt der Jugendlichen verknüpft sind.
\section{Guter Unterricht – Theorien und Qualitätsmerkmale}
\subsection{Zehn Merkmale guten Unterrichts (Hilbert Meyer)}
Der renommierte Pädagoge Hilbert Meyer, dessen handlungsorientierter Ansatz unser Projekt maßgeblich beeinflusst, entwickelte in seinem Werk "Was ist guter Unterricht?" (2004) einen Kriterien-Mix, der sowohl Ergebnisse empirischer Studien als auch seine normative Sicht auf qualitativ hochwertigen Unterricht integriert. Diese normative Dimension betont die Frage nach dem "Sollen", insbesondere in Bezug auf Handlungsweisen, Werte und anzustrebende Ziele, was eine Reflexion im fachlichen Kontext unerlässlich macht. Die zehn Merkmale nach Meyer umfassen:
\begin{enumerate}
    \item \textbf{Klare Strukturierung des Unterrichts}
    \item \textbf{Hoher Anteil echter Lernzeit}
    \item \textbf{Lernförderliches Klima}
    \item \textbf{Inhaltliche Klarheit}
    \item \textbf{Sinnstiftendes Kommunizieren}
    \item \textbf{Methodenvielfalt}
    \item \textbf{Individuelles Fördern}
    \item \textbf{Intelligentes Üben}
    \item \textbf{Transparente Leistungserwartungen}
    \item \textbf{Vorbereitete Umgebung}
\end{enumerate}
\subsection{Kriterien für guten Unterricht (Hans Haenisch)}
Die von Hans Haenisch (2002) formulierten "Kriterien zu gutem Unterricht" fokussieren auf übergreifende Aspekte des Lehrens und Lernens und basieren auf empirischen Studien, die den Zusammenhang zwischen Unterrichtsprozessen und Schülerleistungen untersuchen. Zu diesen Kriterien gehören unter anderem:
\begin{enumerate}
    \item \textbf{Den Unterricht curricular klar ausrichten}
    \item \textbf{Orientierung geben}
    \item \textbf{Die aktive Beteiligung verstärken und Lerngelegenheiten bewusst gestalten}
    \item \textbf{Das bisherige Wissen berücksichtigen und entsprechend umstrukturieren}
    \item \textbf{Lernstrategien zeigen}
    \item \textbf{Gelegenheit bieten, das Gelernte zu üben und anzuwenden}
    \item \textbf{Aktivitäten und Lernfortschritte sorgfältig beobachten, kontrollieren, analysieren und Rückmeldungen geben}
    \item \textbf{Phasen kooperativen Lernens systematisch einbauen}
    \item \textbf{Für einen lernförderlichen Unterrichtskontext sorgen}
\end{enumerate}
\subsection{Unterrichtsqualität nach Andreas Helmke}
Andreas Helmke, ein Vertreter der empirischen Erziehungswissenschaft, präsentierte Merkmale der Unterrichtsqualität, die auf umfangreicher Forschung basieren. Dazu zählen:
\begin{itemize}
    \item \textbf{Strukturiertheit, Klarheit, Verständlichkeit}
    \item \textbf{Effiziente Klassenführung und Zeitnutzung}
    \item \textbf{Lernförderliches Unterrichtsklima}
    \item \textbf{Effiziente Klassenführung und Zeitnutzung}
    \item \textbf{Lernförderliches Unterrichtsklima}
    \item \textbf{Ziel-, Wirkungs- und Kompetenzorientierung}
    \item \textbf{Schülerorientierung, Unterstützung}
    \item \textbf{Angemessene Variation von Methoden und Sozialformen}
    \item \textbf{Aktivierung: Förderung aktiven, selbstständigen Lernens}
    \item \textbf{Konsolidierung, Sicherung, Intelligentes Üben}
    \item \textbf{Vielfältige Motivierung}
    \item \textbf{Passung: Umgang mit heterogenen Lernvoraussetzungen}
\end{itemize}
\subsection{Merkmale wirksamen Unterrichts (Marcus Pietsch)}
Marcus Pietsch betont, dass effektiver Unterricht weniger durch einzelne Merkmale als vielmehr durch das gelungene Zusammenspiel verschiedener Qualitätsaspekte ("Orchestrierung" oder "Choreografie" des Unterrichts) geprägt ist. Er verweist auf die Bedeutung tiefenstruktureller Merkmale für die Lernentwicklung der Schülerinnen und Schüler.
\subsection{Erkenntnisse der Schul- und Unterrichtsforschung (Hartmut Ditton)}
Die Forschung von Hartmut Ditton hebt zentrale Qualitätsdimensionen für den Unterricht hervor, die in unserem Projekt Beachtung finden:

Qualität (Quality): Struktur und Strukturiertheit, Klarheit, Methodenvariabilität, angemessenes Tempo, Mediennutzung, Übungsintensität, Stoffumfang, Leistungserwartungen, Motivierung, bedeutsame Inhalte, bekannte Ziele, Vermeidung von Leistungsangst, Interesse und Neugier wecken, Bekräftigung, positives Sozialklima.

Angemessenheit (Appropriateness): Schwierigkeitsgrad, Adaptivität, diagnostische Sensibilität, individuelle Unterstützung, Differenzierung, Förderungsorientierung.

Unterrichtszeit (Time): Verfügbare Zeit, Lerngelegenheiten, genutzte Lernzeit, Inhaltsorientierung, Klassenmanagement.
\section{Das Didaktische Sechseck (Hilbert Meyer) als Analyseinstrument}
Ein zentrales Analyseinstrument für die Konzeption und Reflexion unseres kompetenzorientierten Scratch-Unterrichts bildet das Didaktische Sechseck nach Hilbert Meyer. Dieses Modell umfasst sechs Kernelemente, die bei der Planung und Durchführung jeder Unterrichtseinheit berücksichtigt werden:
\begin{itemize}
    \item \textbf{Zielstruktur: Welche Kompetenzen und Lernziele sollen die Kinder erreichen?}
    \item \textbf{Inhaltsstruktur: Wie sind die Inhalte didaktisch aufbereitet und sequenziert?}
    \item \textbf{Prozessstruktur: Wie wird der Unterrichtsablauf gestaltet und welche Methoden kommen zum Einsatz?}
    \item \textbf{Handlungsstruktur: Welche Aktivitäten führen die Kinder durch und wie aktiv sind sie beteiligt?}
    \item \textbf{Sozialstruktur: Wie wird das soziale Miteinander im Lernprozess gefördert?}
    \item \textbf{Raumstruktur: Wie ist der Lernraum gestaltet, um das Lernen optimal zu unterstützen?}
\end{itemize}
Die bewusste Auseinandersetzung mit diesen sechs Dimensionen ermöglicht eine strukturierte Planung und Analyse des Unterrichtsgeschehens im Hinblick auf seine Qualität und Effektivität.
\section{Technische Grundlagen: Einführung in Scratch}
\subsection{Was ist Scratch?}
Scratch ist eine visuelle Programmiersprache, die speziell für Kinder und Einsteiger entwickelt wurde. Mit Scratch können Nutzer:innen interaktive Geschichten, Spiele und Animationen erstellen, ohne komplexen Code schreiben zu müssen. Die Plattform fördert spielerisches Lernen und ermöglicht es, grundlegende Programmierkonzepte auf kreative Weise zu verstehen.
Scratch wurde vom MIT Media Lab entwickelt und ist kostenlos nutzbar. Es eignet sich besonders für Kinder ab etwa 8 Jahren, aber auch ältere Lernende können damit die Grundlagen der Programmierung entdecken.
\subsection{Grundprinzipien der Programmierumgebung}
Scratch arbeitet mit visuellen Blöcken und einem Drag-and-Drop-System, das den Einstieg in die Programmierung vereinfacht:

Visuelle Blöcke:
Befehle werden als farbige Code-Blöcke dargestellt, die nach Kategorien (z.B. Bewegung, Ereignisse, Steuerung) sortiert sind.
Logische Abfolgen werden durch das Zusammenstecken der Blöcke gebildet – ähnlich wie Puzzle-Teile.
Drag-and-Drop-Oberfläche:
Nutzer:innen ziehen die Blöcke per Maus oder Touch in den Skriptbereich und fügen sie zusammen.
Dies vermeidet Tippfehler und macht Programmierung intuitiv zugänglich.

\subsection{Arbeiten mit Scratch aus Sicht der Kinder}
\subsection{Typische Projektbeispiele}
Mit Scratch lassen sich vielfältige Projekte umsetzen, z.B.:

Animationen:
Einfache Bewegungsabläufe (z.B. eine tanzende Figur) oder kurze Geschichten.
Spiele:
Klassiker wie „Pong“, „Maze“ (Labyrinth) oder eigene Jump-&-Run-Spiele.
Interaktive Anwendungen:
Digitale Grußkarten, Quizze oder Simulationen (z.B. ein virtuelles Haustier).

\subsection{Altersgerechte Funktionen und Anwendungsmöglichkeiten}
Scratch ist auf junge Nutzer:innen zugeschnitten:

Figuren (Sprites) & Hintergründe:
Große Bibliothek an vorgefertigten Charakteren und Szenen.
Eigene Zeichnungen oder Fotos können importiert werden.
Sounds & Musik:
Einfache Audiobearbeitung (z.B. Töne aufnehmen, Effekte hinzufügen).
Einfache Logik:
Grundlegende Programmierkonzepte wie Schleifen („wiederhole“), Bedingungen („falls… dann“) oder Variablen werden kindgerecht vermittelt.
